\begin{exsol@exercise}{15.-39}
    \begin{center}
\begin{flushleft}\textbf{\large \hfill Workshop 13C, Submitted by: }\end{flushleft}

\fbox{\parbox{14cm}{
\vspace{4mm}
Name: \underline{\phantom{xxxxxxxxxxxxxxxxxxxxxxxx}} Signature: \underline{\phantom{xxxxxxxxxxxxxxxxxxxxxxxx}}

\vspace{4mm}
Name: \underline{\phantom{xxxxxxxxxxxxxxxxxxxxxxxx}} Signature: \underline{\phantom{xxxxxxxxxxxxxxxxxxxxxxxx}}

\vspace{4mm}
Name: \underline{\phantom{xxxxxxxxxxxxxxxxxxxxxxxx}} Signature: \underline{\phantom{xxxxxxxxxxxxxxxxxxxxxxxx}}

\vspace{4mm}
Name: \underline{\phantom{xxxxxxxxxxxxxxxxxxxxxxxx}} Signature: \underline{\phantom{xxxxxxxxxxxxxxxxxxxxxxxx}}
 }}
\end{center}

Making the Cut

The following table shows how many prospective students of a sample of 200 students were admitted to WMU last year, and if the student's GPA was below 2.8 or not. Out of the sample, 60 applicants were rejected and 140 were accepted.
			
\begin{tabular}{@{} lccc @{}} \hline				
 	& \multicolumn{2}{c}{Rejected from WMU} \\ \hline				
GPA Below 2.8 &	Yes &	No &	Total  \\ \hline
Yes &	45 &	20 &	65 \\
No &	15 &	120 &	135 \\ \hline
Total &	60 &	140 &	200 \\ \hline
\end{tabular}
				
\begin{enumerate}
  \item Estimate the risk ratio of being rejected from WMU between having below a 2.8 GPA and not having below a 2.8 GPA.
  \item Calculate a 95\% confidence interval for the risk ratio of being rejected from WMU between having below a 2.8 GPA and not having below a 2.8 GPA.
\end{enumerate}

\end{exsol@exercise}
