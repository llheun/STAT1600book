\begin{exsol@exercise}{16.-33}
    \begin{center}
\begin{flushleft}\textbf{\large \hfill Workshop 22C, Submitted by: }\end{flushleft}

\fbox{\parbox{14cm}{
\vspace{4mm}
Name: \underline{\phantom{xxxxxxxxxxxxxxxxxxxxxxxx}} Signature: \underline{\phantom{xxxxxxxxxxxxxxxxxxxxxxxx}}

\vspace{4mm}
Name: \underline{\phantom{xxxxxxxxxxxxxxxxxxxxxxxx}} Signature: \underline{\phantom{xxxxxxxxxxxxxxxxxxxxxxxx}}

\vspace{4mm}
Name: \underline{\phantom{xxxxxxxxxxxxxxxxxxxxxxxx}} Signature: \underline{\phantom{xxxxxxxxxxxxxxxxxxxxxxxx}}

\vspace{4mm}
Name: \underline{\phantom{xxxxxxxxxxxxxxxxxxxxxxxx}} Signature: \underline{\phantom{xxxxxxxxxxxxxxxxxxxxxxxx}}
 }}
\end{center}

Session musicians in a studio are worried about their tuning. To get just the right mix, they decide to measure the frequency of the notes they play while recording. They begin by measuring frequency in Hertz at the note ``middle C'' (0 in the table), and work their way up by regular intervals (whole steps) from there. From ``middle C'' through the first four whole step intervals, they produce the following data:

\begin{tabular}{@{} ccc @{}} \hline
Intervals &	Frequency (Hz) \\ \hline
0&	261.626 \\
1&	293.665 \\
2&	329.628 \\
3&	369.994 \\
4&	415.305 \\ \hline
\end{tabular}

\begin{enumerate}
  \item Write the full regression equation, using Intervals as explanatory and Frequency as response. The standard deviations of Intervals and Frequency are, respectively, 1.581 and 60.807. The correlation coefficient is 0.998
  \item	Interpret the slope of this regression line in context.
  \item	The session musicians take another measurement and find that the fifth whole step interval up from middle C, the A sharp, has an actual frequency of 466.164 Hz. Using your line from part a, calculate the residual for the predicted fifth interval from middle C.
  \item	Calculate a 95\% confidence interval for the slope of the line. Is the slope statistically significant?

\end{enumerate}

\end{exsol@exercise}
