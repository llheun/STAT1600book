\begin{exsol@exercise}{15.-24}
    \begin{center}
\begin{flushleft}\textbf{\large \hfill Workshop 8C, Submitted by: }\end{flushleft}

\fbox{\parbox{14cm}{
\vspace{4mm}
Name: \underline{\phantom{xxxxxxxxxxxxxxxxxxxxxxxx}} Signature: \underline{\phantom{xxxxxxxxxxxxxxxxxxxxxxxx}}

\vspace{4mm}
Name: \underline{\phantom{xxxxxxxxxxxxxxxxxxxxxxxx}} Signature: \underline{\phantom{xxxxxxxxxxxxxxxxxxxxxxxx}}

\vspace{4mm}
Name: \underline{\phantom{xxxxxxxxxxxxxxxxxxxxxxxx}} Signature: \underline{\phantom{xxxxxxxxxxxxxxxxxxxxxxxx}}

\vspace{4mm}
Name: \underline{\phantom{xxxxxxxxxxxxxxxxxxxxxxxx}} Signature: \underline{\phantom{xxxxxxxxxxxxxxxxxxxxxxxx}}
 }}
\end{center}

The Bureau of Labor Statistics reports that in May 2016 that 10.5\% of people who work in the Kalamazoo metropolitan area are employed in production occupations (i.e., making things). Suppose that a random sample of 5 Kalamazoo workers is taken. Let $X$ be a random variable which denotes the number of workers in the sample who are employed in production occupations.

\begin{enumerate}
  \item What are the possible values which X can take?
  \item Find the probability that exactly 2 of the 5 workers in the sample will be employed in production occupations.
  \item Find the probability that at least 3 of the 5 workers in the sample will be employed in production occupations.
  \item Find the probability that at most 1 of the 5 workers in the sample will be employed in production occupations.
\end{enumerate}

\end{exsol@exercise}
