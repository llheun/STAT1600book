<<<<<<< HEAD
\begin{exsol@exercise}{15.-39}
    \begin{center}
\begin{flushleft}\textbf{\large \hfill Workshop 13C, Submitted by: }\end{flushleft}
=======
\begin{exsol@exercise}{15.-24}
    \begin{center}
\begin{flushleft}\textbf{\large \hfill Workshop 8C, Submitted by: }\end{flushleft}
>>>>>>> 3d6c258c135c5baf35b6b4ed045ea5ef439c4c51

\fbox{\parbox{14cm}{
\vspace{4mm}
Name: \underline{\phantom{xxxxxxxxxxxxxxxxxxxxxxxx}} Signature: \underline{\phantom{xxxxxxxxxxxxxxxxxxxxxxxx}}

\vspace{4mm}
Name: \underline{\phantom{xxxxxxxxxxxxxxxxxxxxxxxx}} Signature: \underline{\phantom{xxxxxxxxxxxxxxxxxxxxxxxx}}

\vspace{4mm}
Name: \underline{\phantom{xxxxxxxxxxxxxxxxxxxxxxxx}} Signature: \underline{\phantom{xxxxxxxxxxxxxxxxxxxxxxxx}}

\vspace{4mm}
Name: \underline{\phantom{xxxxxxxxxxxxxxxxxxxxxxxx}} Signature: \underline{\phantom{xxxxxxxxxxxxxxxxxxxxxxxx}}
 }}
\end{center}

<<<<<<< HEAD
Making the Cut
=======
The Bureau of Labor Statistics reports that in May 2016 that 10.5\% of people who work in the Kalamazoo metropolitan area are employed in production occupations (i.e., making things). Suppose that a random sample of 5 Kalamazoo workers is taken. Let $X$ be a random variable which denotes the number of workers in the sample who are employed in production occupations.
>>>>>>> 3d6c258c135c5baf35b6b4ed045ea5ef439c4c51

The following table shows how many prospective students of a sample of 200 students were admitted to WMU last year, and if the student's GPA was below 2.8 or not. Out of the sample, 60 applicants were rejected and 140 were accepted.
			
\begin{tabular}{@{} lccc @{}} \hline				
 	& \multicolumn{2}{c}{Rejected from WMU} \\ \hline				
GPA Below 2.8 &	Yes &	No &	Total  \\ \hline
Yes &	45 &	20 &	65 \\
No &	15 &	120 &	135 \\ \hline
Total &	60 &	140 &	200 \\ \hline
\end{tabular}
				
\begin{enumerate}
<<<<<<< HEAD
  \item Estimate the risk ratio of being rejected from WMU between having below a 2.8 GPA and not having below a 2.8 GPA.
  \item Calculate a 95\% confidence interval for the risk ratio of being rejected from WMU between having below a 2.8 GPA and not having below a 2.8 GPA.
=======
  \item What are the possible values which X can take?
  \item Find the probability that exactly 2 of the 5 workers in the sample will be employed in production occupations.
  \item Find the probability that at least 3 of the 5 workers in the sample will be employed in production occupations.
  \item Find the probability that at most 1 of the 5 workers in the sample will be employed in production occupations.
>>>>>>> 3d6c258c135c5baf35b6b4ed045ea5ef439c4c51
\end{enumerate}

\end{exsol@exercise}
