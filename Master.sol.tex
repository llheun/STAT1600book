\setcounter{chapter}{1}\chapter{Knowledge and Data}
\begin{exsol@solution}{1.4-1}
  What might be wrong about these headlines?
  a.	A study proclaims: ``Slightly overweight people live longer than thin people.'' ``Slightly overweight people live longer than thin people.''  It is a fallacy of correlation equals causation.  It statement implies that being slightly overweight causes people to live longer when compared to thin people.  We would want to dig deeper into the study to see what confounding variables may be present: did we take into account health of the individuals?  Perhaps we have some individuals in poor health leading them to be thin and shortening their lives.

\end{exsol@solution}
\begin{exsol@solution}{1.4-2}
  A study proclaims: ``Mountain Dew is the beverage of choice in mainstream America.''  This sample was taken at a local shopping mall in the U.S.
\end{exsol@solution}
\begin{exsol@solution}{1.4-3}
xxx

\end{exsol@solution}
\begin{exsol@solution}{1.4-4}
	   YYYY
\end{exsol@solution}
\setcounter{chapter}{2}\chapter{Data Presentation}
\begin{exsol@solution}{2.6-1}
    \begin{enumerate}
	  \item The scores on exam one for Stat 1600:  numeric, ratio
    \item Marital status: categorical, nominal
    \item Annual income: numeric, ratio
    \item Social Security Number: categorical, nominal
    \item Cumulative GPA: numeric, interval
    \item Academic level (freshman, sophomore, junior, senior, other): categorical, ordinal
    \item Quality (poor, fair, good, excellent): categorical, ordinal
    \item Height (short, average, tall): categorical, ordinal
    \item Age (years): numeric: ratio
    \item Grade $(A, B, C, \dots)$: categorical, ordinal
    \item Color: categorical, nominal
    \item Rating of eight local plays (poor, fair, good, excellent): categorical, ordinal
    \item Times required for mechanics to do a tune-up: numeric, interval
  	\end{enumerate}
\end{exsol@solution}
\begin{exsol@solution}{2.6-3}
2.5-7.1: Left-skewed

2.5-7.2: Symmetric

2.5-7.3: Left-skewed

2.5-7.4: Right-skewed

2.5-7.5: Symmetric

\end{exsol@solution}
\begin{exsol@solution}{2.6-4}

Compute relative frequencies:

\begin{tabular}{@{} lcc @{}} \hline
Interval  &  Frequency &	Relative Frequency \\
$(20, 25]$ 	&     4 & 4  \\
$(25, 30]$ 	&    11 & 11 \\
$(30, 35]$ 	&    23 & 23 \\
$(35, 40]$ 	&    31 & 31 \\
$(40, 45]$ 	&    15 & 15 \\
$(45, 50]$ 	&    10 & 10 \\
$(50, 55]$ 	&     6 & 6 \\ \hline
\end{tabular}


\end{exsol@solution}
\setcounter{chapter}{3}\chapter{Location and Spread}
\begin{exsol@solution}{3.4-1}

	Compute the mean, median, $\dots$ for carbon monoxide:
  \begin{enumerate}
  \item mean = 12.5
  \item median = 13.0
  \item Trimmed mean = 12.7
  \item SD = 4.74
  \item Mean = 0.0125 and standard deviation 0.00474 (in grams)
\end{exsol@solution}
\begin{exsol@solution}{3.4-3}


\begin{enumerate}
\item The range is from 13 to 59
\item the mean is 46.7
\item The median is 49.5
\item There is no mode since there are no duplicates.
\item Removing the smallest value changes the mean to 50.4.  The new value is closer to median because 13 is an outlier.
\end{enumerate}
\end{exsol@solution}
\begin{exsol@solution}{3.4-4}


\begin{enumerate}
\item the mean is 43.4166667
\item the median is 39
\item Decide if its symmetric, skewed to the right or to the left: the data is right skewed
\item Decide which measure of center provides the most relevant information about the distribution? Why?  The median is the most relevant information because the data is right skewed.
\end{enumerate}

\end{exsol@solution}
\setcounter{chapter}{4}\chapter{Threats to Valid Comparisons}
\begin{exsol@solution}{4.4-1}
	  $H_0: \mu = 6.2$ vs. $H_A: \mu \neq 6.2$
\end{exsol@solution}
\begin{exsol@solution}{4.4-3}

    From the distribution of $t$, using row df = 24 and column 0.025, CV(t) = 2.064.
\end{exsol@solution}
\setcounter{chapter}{5}\chapter{Study Designs}
\begin{exsol@solution}{5.5-1}
	  $H_0: \mu_1 = \mu_2$ vs. $H_0: \mu_1 \neq \mu_2$
\end{exsol@solution}
\begin{exsol@solution}{5.5-3}
The test statistics is $|-1.66|$ and significance level is 1.8125.  Since the test statistic is less than the significance level, fail to reject $H_0$.
\end{exsol@solution}
\begin{exsol@solution}{5.5-5}

    Standard Normal (z) distribution, $Z = \pm 1.96$.
\end{exsol@solution}
\begin{exsol@solution}{5.5-7}
    $H_0: P_1 = P_2$ vs. $H_A: P_1 > P_2$
\end{exsol@solution}
\setcounter{chapter}{6}\chapter{The Normal Distribution}
\begin{exsol@solution}{6.6-1}

\begin{enumerate}
	\item The proportion of cell phone users are on their phones between 1 hour
and 3 hours per day is 0.6057221
  \item Just to be safe, suppose you decide to be in the 5th percentile of
cell phone users in terms of monthly usage.  How much time can you spend on your phone per day? You decide to spend no more than 35.4396606 minutes on your phone per day.
	\end{enumerate}
\end{exsol@solution}
\begin{exsol@solution}{6.6-2}


\begin{enumerate}
	\item The percentage of watch batteries last more than 6 months is $P[ X > 6] = 0.998067$.
	\item What is the life span of a watch battery which lasts longer than 60\% of all batteries?  $P[X > a] = .60$. Now solve for a = 21.7198761.
	\item What proportion of watch batteries last shorter than 2 years or longer
than 3 1/2 years (42 months)?
$P[X < 24] + P[X > 42] = .5 + 0.0227501 = 0.5227501$
 \end{enumerate}

\end{exsol@solution}
\begin{exsol@solution}{6.6-3}


    The standard deviation is $SD = \frac{35 - 4}{4} = 7.75$
\end{exsol@solution}
\begin{exsol@solution}{6.6-4}

\begin{enumerate}
\item The probability the value is greater than 6 is $P[X > 6] = 1$
\item The probability the value is less than 12 is $P[X < 12] = \ensuremath{3.3976731\times 10^{-6}}$
\item The probability the value is between 6 and 12 is $P[ 6 < X < 12] = 0.9999966$
\item 33\% is above a, i.e., $P[X > a] = .33; a = 10.8798263$
\item 33\% is below b, i.e., $P[X < b] = .33; b = 9.1201737$
\end{enumerate}
\end{exsol@solution}
\begin{exsol@solution}{6.6-5}

\begin{enumerate}
\item The probability that the stock price is between 39.88 and 46.01 is $P[39.88 < X < 46.01] = 0.9395927$.
\item The probability that the stock price is above 40 is $P[X > 40] = 0.9327388$.
\item The probability that the stock price is below 40 si $P[ x < 40] = 0.0672612$.
\end{enumerate}
\end{exsol@solution}
\begin{exsol@solution}{6.6-6}

\begin{enumerate}
\item The proportion of adult female heights is below 72 is $P[ X < 72] = 0.999683$.
\item 25\% of adult females are greater than $P[ X > a] = .25$ where a = 65.4187754
\end{enumerate}
\end{exsol@solution}
\begin{exsol@solution}{6.6-7}
\begin{enumerate}
\item The area to the left of 0.0 is.5000
\item The area to the left of 0.2 is.5793
\item The area to the left of 0.25 is.5987
\item The area to the left of 2.25 is.9878
\end{enumerate}
\end{exsol@solution}
\setcounter{chapter}{7}\chapter{The Binomial Distribution}
\begin{exsol@solution}{7.7-1}

		\begin{enumerate}
	  \item The mean and standard deviation are 5 and 1.5811388, respectively.
    \item The probability that there are more than 5 successes is $P[ X > 5 ] = 0.3769531$.
    \item The probability that there are fewer than 5 successes is $P[ X < 5 ] = 0.3769531$.
    \item The probability that there  are between 1 and 3 successes  is $P[ 1 \le X \le 5 ] = 0.1708984$.
	  \end{enumerate}
\end{exsol@solution}
\begin{exsol@solution}{7.7-2}

\begin{enumerate}
\item The probability that at least 2 questions are correct is $P[X \ge 2] = 0.6241904$.
\item The probability that at most 2 questions arecorrect is $P[X \le 2] = 0.6777995$.
\item The probability that there  will be between 1 and 3 questions is $P[1 \le X \le 3] = 0.7717519$.
\end{enumerate}
\end{exsol@solution}
\begin{exsol@solution}{7.7-3}

\begin{enumerate}
\item The probability that at least 100 people are Apple users is $P[X \ge 100] = 0.3070581$.
\item The probability that at most 100 people are Apple users is $P[X \le 100] = 0.7302684$.
\item The probability that between 80 and \\ 120 people are Apple users is $P[ 80 \le X \le 120] = 0.9572505$.
\end{enumerate}
\end{exsol@solution}
\begin{exsol@solution}{7.7-4}


\begin{enumerate}
\item The probability of rolling a 6 no more than 3 times is $P[X \le 3] = 0.9302722$.
\item The probability that no less than 3 times is $P[X \ge 3] = 0.2247732$.
\end{enumerate}
\end{exsol@solution}
\begin{exsol@solution}{7.7-5}

\begin{enumerate}
\item The probability that he gets at least 7 hits is $P[X \ge 7] = 0.0048184$.
\item The probability that he gets at most 1 hit is $P[X \le 1] = 0.1461307$.
\item The probability that he gets between 4 and 6 hits is $P[4 \le X \le 6] = 0.3245785$.
\end{enumerate}

\end{exsol@solution}
\begin{exsol@solution}{7.7-6}

The expected value is 1.27 and SD is 1.1125385

\end{exsol@solution}
\begin{exsol@solution}{7.7-7}

The expected value is 8 and SD is 2.7712813
\end{exsol@solution}
\setcounter{chapter}{8}\chapter{Sampling Distribution of the Proportion}
\begin{exsol@solution}{8.6-1}

$P[ X \ge 32 ] = 0.0760321
\end{exsol@solution}
\begin{exsol@solution}{8.6-2}

\begin{enumerate}
\item The estimate of the population proportion is 0.2
\item The standard error of this estimate is 0.04
\item The 95\% margin of error is 0.0784
\item The 95\% confidence interval is $0.2 \pm 0.0784$
\end{enumerate}
\end{exsol@solution}
\begin{exsol@solution}{8.6-3}

The standard error of this estimate is 0.04

\end{exsol@solution}
\begin{exsol@solution}{8.6-4}

\begin{enumerate}
\item	The estimate of the population proportion is \\ $\hat{p} = 0.066$
\item	The standard error of this estimate is \\ $SE_{\hat{p}} = 0.0111035$
\item	The 95\% margin of error is $M_{\hat{p}} = 0.0217629$
\item	The 95\% confidence interval is \\ $0.066 \pm 0.0217629$
\end{enumerate}
\end{exsol@solution}
\begin{exsol@solution}{8.6-5}

\begin{enumerate}
\item	The estimate of the population proportion is \\ $\hat{p} = 0.2583333$
\item	The standard error of this estimate is \\ $SE_{\hat{p}} = 0.039958$
\item	The 95\% confidence interval is \\ $0.2583333 \pm 0.0783177$
\end{enumerate}
\end{exsol@solution}
\begin{exsol@solution}{8.6-6}

\begin{enumerate}
\item	The estimate of the population proportion is \\ $\hat{p} = 0.3182725$
\item	The standard error of this estimate is \\ $SE_{\hat{p}} = 0.0062747$
\item The margin of error the estimate is \\ $M_{\hat{p}} = 0.0122983$
\item	The 95\% confidence interval is \\ $0.3182725 \pm 0.0122983$
\end{enumerate}
\end{exsol@solution}
\begin{exsol@solution}{8.6-7}

\begin{enumerate}
\item	The estimate of the population proportion is \\ $\hat{p} = 0.3$
\item	The standard error of this estimate is \\ $SE_{\hat{p}} = 0.010247$
\item The margin of error the estimate is \\ $M_{\hat{p}} = 0.020084$
\item	The 95\% confidence interval is \\ $0.3 \pm 0.020084$
\item	The 95\% confidence interval is \\ $0.3 \pm 0.020084$
\end{enumerate}
\end{exsol@solution}
\setcounter{chapter}{9}\chapter{Comparing Two Proportions }
\begin{exsol@solution}{9.5-1}


\begin{enumerate}
\item The difference in percentage of  drug  use \\ between smokers and nonsmokers is 35.0.
\item Calculate a standard error for your estimate in (1).
\item Calculate a 95\% confidence interval for the difference in percentage of drug use \\ between smokers and nonsmokers.
\item Estimate the risk ratio of drug use \\ between smokers and nonsmokers.
\item Calculate a standard error for the natural log of your estimate in 4.
\item Calculate a 95\% confidence interval for the risk ratio of drug use between smokers
\item Estimate the odds ratio of drug use \\ between smokers and nonsmokers.
\item Calculate a standard error for the natural log of your estimate in 7.
\item Calculate a 95\% confidence interval for the odds ratio of drug use between smokers and nonsmokers.
\item Interpret the above confidence intervals in parts 3, 6, and 9. Which are significant, and which are not? Why or why not?
\end{enumerate}

\end{exsol@solution}
\begin{exsol@solution}{9.5-2}
		  \begin{enumerate}
	  \item Estimate the difference in percentage of Americans who supported the federal tax on cigarettes between smokers and non-smokers.
    \item Calculate a standard error for your estimate in (1).
    \item Calculate a 95\% confidence interval for the difference in percentage of Americans who supported the federal tax on cigarettes between smokers and non-smokers.
    \item Estimate the risk ratio of Americans who supported the federal tax on cigarettes between smokers and non-smokers.
    \item Estimate the odds ratio of Americans who supported the federal tax on cigarettes between smokers and non-smokers.
	  \end{enumerate}

\end{exsol@solution}
\begin{exsol@solution}{9.5-3}
   The critical value is
[1] 1.984217

\end{exsol@solution}
\setcounter{chapter}{10}\chapter{Sampling Distribution of the Mean}
\begin{exsol@solution}{10.6-1}
\end{exsol@solution}
\begin{exsol@solution}{10.6-2}
\end{exsol@solution}
\begin{exsol@solution}{10.6-3}
\end{exsol@solution}
\begin{exsol@solution}{10.6-4}
\end{exsol@solution}
\begin{exsol@solution}{10.6-5}
\end{exsol@solution}
\begin{exsol@solution}{10.6-6}
\end{exsol@solution}
\setcounter{chapter}{11}\chapter{Comparing Two Means}
\begin{exsol@solution}{11.5-1}

\end{exsol@solution}
\begin{exsol@solution}{11.5-2}
\end{exsol@solution}
\begin{exsol@solution}{11.5-3}
\end{exsol@solution}
\begin{exsol@solution}{11.5-4}

\end{exsol@solution}
\begin{exsol@solution}{11.5-5}
\end{exsol@solution}
\begin{exsol@solution}{11.5-6}
\end{exsol@solution}
\setcounter{chapter}{12}\chapter{Categorical Variables: Association or Independence}
\begin{exsol@solution}{12.3-1}
\end{exsol@solution}
\begin{exsol@solution}{12.3-2}
\end{exsol@solution}
\begin{exsol@solution}{12.3-3}
\end{exsol@solution}
\begin{exsol@solution}{12.3-4}
\end{exsol@solution}
\begin{exsol@solution}{12.3-5}
\end{exsol@solution}
\setcounter{chapter}{13}\chapter{Correlation}
\begin{exsol@solution}{13.3-1}


13.5-1.1: $r = 0.9527 $

13.5-1.2: same, $r = 0.9527 $

13.5-1.3: same, $r = 0.9527 $

13.5-1.4: same, $r = 0.9527 $

13.5-1.5: same but the sign changed, $r = -0.9527 $

\end{exsol@solution}
\begin{exsol@solution}{13.3-2}
\end{exsol@solution}
\begin{exsol@solution}{13.3-3}
\end{exsol@solution}
\begin{exsol@solution}{13.3-4}
\end{exsol@solution}
\begin{exsol@solution}{13.3-5}
\end{exsol@solution}
\begin{exsol@solution}{13.3-6}
\end{exsol@solution}
\setcounter{chapter}{14}\chapter{Linear Regression}
\begin{exsol@solution}{14.7-1}
\begin{enumerate}
  \item Calculate the regression line for predicting Y from X.
  \item Draw the scatterplot with an overlaid regression line.
  \item Add 5 to Y, so the new values are 5, 8, 15, 6, 20.  Calculate the new regression line.
  \item Multiply Y by 5, so the new values are 0, 15, 50, 5, 75.  Calculate the new regression line.
\end{enumerate}

\end{exsol@solution}
\begin{exsol@solution}{14.7-2}
\end{exsol@solution}
\begin{exsol@solution}{14.7-3}

\end{exsol@solution}
\setcounter{chapter}{15}\chapter{Workshops}
\begin{exsol@solution}{15.-1}

  \begin{enumerate}
  \item Construct a relative frequency table for total student majors (column above).
  \item	The highest percentage of students fall under what major?
  \item	What percentage of students are Art majors?
  \item	What percentage of students’ majors fall in both Sociology and Psychology?
  \item	How does the percentage of students who are communications majors in Class A compare to the percentage of communication majors overall for total students?
\end{enumerate}
\end{exsol@solution}
\begin{exsol@solution}{15.-2}
  \begin{enumerate}
  \item	Using the column `RF' above, construct a relative frequency table for student majors in the Web Class only.
  \item	The highest percentage of students fall under what major?
  \item	What percentage of students are Music majors?
  \item	What percentage of students are found in the majors of Psychology, Social Work and Sociology (combined)?
  \item	How does the percentage of students who are education majors in the Web class compare to the percentage of education majors overall for total students?  Which is higher?
\end{enumerate}
\end{exsol@solution}
\begin{exsol@solution}{15.-3}
  \begin{enumerate}
  \item	Using the column `RF' above, construct a relative frequency table for student majors in the Web Class only.
  \item	The highest percentage of students fall under what major?
  \item	What percentage of students are Music majors?
  \item	What percentage of students are found in the majors of Psychology, Social Work and Sociology (combined)?
  \item	How does the percentage of students who are education majors in the Web class compare to the percentage of education majors overall for total students?  Which is higher?
\end{enumerate}
\end{exsol@solution}
\begin{exsol@solution}{15.-4}
  \begin{enumerate}
  \item Fill in the frequency and relative frequency table above
  \item  Create a bar chart and pie chart for the above data
  \item  What percentage of students got a grade of `A'?
  \item  Looking at the bar chart in part a, identify the shape of the data (symmetric, right-skewed, left-skewed)?
\end{enumerate}
\end{exsol@solution}
\begin{exsol@solution}{15.-5}
  \begin{enumerate}
  \item Fill in the frequency and relative frequency above.
  \item	Complete a stem and leaf plot with the stem representing the 10’s place and using 3-9.
  \item Draw a histogram for the test scores using the intervals in the relative frequency table above.
  \item	What percentage of students had scores of 59 or less?
\end{enumerate}

\end{exsol@solution}
\begin{exsol@solution}{15.-6}
\begin{enumerate}
  \item	Fill in the frequency and relative frequency above.
  \item	Complete a stem and leaf plot with the stem representing the 10's place and using 0-2.
  \item	Draw a histogram for the absences using the intervals in the relative frequency table above.
  \item	What percentage of students had number of absences less than 20?
\end{enumerate}

\end{exsol@solution}
\begin{exsol@solution}{15.-7}
\begin{enumerate}
  \item	Fill in the frequency and relative frequency above.
  \item	Complete a stem and leaf plot with the stem representing the 10's place and using 0-2.
  \item	Draw a histogram for the absences using the intervals in the relative frequency table above.
  \item	What percentage of students had number of absences less than 20?
\end{enumerate}

\end{exsol@solution}
\begin{exsol@solution}{15.-8}
\begin{enumerate}
  \item	Fill in the frequency and relative frequency above.
  \item	Complete a stem and leaf plot with the stem representing the 10's place and using 0-2.
  \item	Draw a histogram for the absences using the intervals in the relative frequency table above.
  \item	What percentage of students had number of absences less than 20?
\end{enumerate}

\end{exsol@solution}
\begin{exsol@solution}{15.-9}
\end{exsol@solution}
\begin{exsol@solution}{15.-10}
\end{exsol@solution}
\begin{exsol@solution}{15.-11}
\end{exsol@solution}
\begin{exsol@solution}{15.-12}
\end{exsol@solution}
\begin{exsol@solution}{15.-13}
\end{exsol@solution}
\begin{exsol@solution}{15.-14}
\end{exsol@solution}
\begin{exsol@solution}{15.-15}
\end{exsol@solution}
\begin{exsol@solution}{15.-16}
\end{exsol@solution}
\begin{exsol@solution}{15.-17}
\end{exsol@solution}
\begin{exsol@solution}{15.-18}
\end{exsol@solution}
\begin{exsol@solution}{15.-19}
\end{exsol@solution}
\begin{exsol@solution}{15.-20}
\end{exsol@solution}
\begin{exsol@solution}{15.-21}
\end{exsol@solution}
\begin{exsol@solution}{15.-22}
\end{exsol@solution}
\begin{exsol@solution}{15.-23}
\end{exsol@solution}
\begin{exsol@solution}{15.-24}
\end{exsol@solution}
\begin{exsol@solution}{15.-25}
\end{exsol@solution}
\begin{exsol@solution}{15.-26}
\end{exsol@solution}
\begin{exsol@solution}{15.-27}
\end{exsol@solution}
\begin{exsol@solution}{15.-28}
\end{exsol@solution}
\begin{exsol@solution}{15.-29}
\end{exsol@solution}
\begin{exsol@solution}{15.-30}
\end{exsol@solution}
\begin{exsol@solution}{15.-31}
\end{exsol@solution}
\begin{exsol@solution}{15.-32}
\end{exsol@solution}
\begin{exsol@solution}{15.-33}
\end{exsol@solution}
\begin{exsol@solution}{15.-34}
\end{exsol@solution}
\begin{exsol@solution}{15.-35}
\end{exsol@solution}
\begin{exsol@solution}{15.-36}
\end{exsol@solution}
\begin{exsol@solution}{15.-37}
\end{exsol@solution}
\begin{exsol@solution}{15.-38}
\end{exsol@solution}
\begin{exsol@solution}{15.-39}
\end{exsol@solution}
\begin{exsol@solution}{15.-40}
\end{exsol@solution}
\begin{exsol@solution}{15.-41}
\end{exsol@solution}
\begin{exsol@solution}{15.-42}
\end{exsol@solution}
\begin{exsol@solution}{15.-43}
\end{exsol@solution}
\begin{exsol@solution}{15.-44}
\end{exsol@solution}
\begin{exsol@solution}{15.-45}
\end{exsol@solution}
\begin{exsol@solution}{15.-46}
\end{exsol@solution}
\begin{exsol@solution}{15.-47}
\end{exsol@solution}
\begin{exsol@solution}{15.-48}
\end{exsol@solution}
\begin{exsol@solution}{15.-49}
\end{exsol@solution}
\begin{exsol@solution}{15.-50}
\end{exsol@solution}
\begin{exsol@solution}{15.-51}
\end{exsol@solution}
